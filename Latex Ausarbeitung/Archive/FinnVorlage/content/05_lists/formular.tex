% % Definition des neuen Befehls für das Einfügen der Abkürzung der Einheit


\DeclareAcroProperty{unit}

\NewAcroTemplate[list]{physics}{%
  \acronymsmapT{%
    \AcroAddRow{%
      \acrowrite{short}%
      &
      \acrowrite{unit}%
      &
      \acrowrite{list}%
      \tabularnewline
    }%
  }%
  \acroheading
  \acropreamble
  \par\noindent
  \setlength\LTleft{0pt}%
  \setlength\LTright{0pt}%
  \begin{longtable}{@{}lll@{\extracolsep{\fill}}l@{}}
    \AcronymTable
  \end{longtable}
}

\DeclareAcronym{f}{
  short = \ensuremath{f} ,
  long  = Frequenz ,
  unit  = \si{\hertz} ,
  tag   = physics
}
\DeclareAcronym{A}{
  short = \ensuremath{A} ,
  long  = Fläche ,
  unit  = \si{\metre^2} ,
  tag   = physics
}
\DeclareAcronym{C}{
  short = \ensuremath{C} ,
  long  = Kapazität ,
  unit  = \si{\farad} ,
  tag   = physics
}
\DeclareAcronym{F}{
  short = \ensuremath{F} ,
  long  = Kraft ,
  unit  = \si{\newton} ,
  tag   = physics
}
