\section{Application Area and Goals} \label{sec:applicationAreaGoals}

Heart diseases are currently one of the highest causes of mortality on earth \citep{nahar2013, kavitha2016, statistischesbundesamt2020}.
Given the successful application of data mining in other sectors e.g. banking and finance or marketing \citep{keles2017} possible applications in the medical industry are plentiful. Yet the healthcare sector is "information rich but knowledge poor" \citep{soni2011}. According to \citet{soni2011} medical datasets provide great potential for data mining to be used in clinical diagnosis.


The aim of this project is the application of data mining methods, more specifically classification methods, to predict whether a patient could suffer from a heart disease or not. The successful application could help doctors and medical staff diagnose patients by automatically analyzing a patient's historical test results and predicting whether there is a heart problem or not. Through this analysis, patients who have been diagnosed with heart disease could receive special treatment. Given the immense stress and long working hours that medical staff are exposed to, a standardized scheme for the evaluation of data could be beneficial. 
In the past such approaches have already been tested and proven to be a good diagnostic option \citep{usharani2011}. \citet{jabbar2013} state that data mining techniques answer several important and critical questions related to healthcare and that they can improve the provision of quality services to patients.

This project report is based on the \say{Heart Disease Dataset} \citep{janosi1988} which was made available by the University of California, Irvine (UCI) Machine Learning Repository under the Creative Commons Attribution 4.0 International License. At the time of writing, the dataset is partially corrupt, so the data is not identical to the original publication. The data is despite its age still relevant given the fact that it consists of results of medical tests. In addition to that is also valid because it is frequently used in contemporary research (see \cite{usharani2011, aha1988, nahar2013}). 