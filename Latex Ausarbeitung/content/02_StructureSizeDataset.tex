%\newpage
\section{Structure and Size of the Dataset} \label{sec:dataUnderstanding}

% minimum 1 page

The data used in this paper comes from four individual datasets collected at different universities (Zurich, Budapest, Long Beach, Cleveland). The distribution of the target variable can be seen in \vref{table:datasets}.
% PLACEHOLDER TABLE
\begin{table}[h]

    \begin{footnotesize}
        \begin{tabular}{|l|l|l|l|}
            \hline
            \textbf{Origin of data set}              & \textbf{\# of instances} & \textbf{Distribution target variable} \\ \hline
            Hungarian Institute of Cardiology, Budapest & 294                      & 106 / 188                             \\ \hline
            University Hospital, Zurich                 & 123                      & 115 / 8                               \\ \hline
            V.A. Medical Center, Long Beach             & 200                      & 149 / 51                              \\ \hline
            Cleveland Clinic Foundation, Cleveland      & 282                      & 125 / 157                             \\ \hline
        \end{tabular}
    \end{footnotesize}
    \begin{center}
        \centering
        Distribution = heart disease / no heart disease
    \end{center}
    \caption{Content of the dataset}
    \label{table:datasets}
\end{table}



For the purposes of this paper, the four datasets are considered as one coherent dataset consisting of 77 attributes (33 numeric, 42 categorical, 1 constant) and a total of 899 measurements.
The attributes can be divided into the following categories:
\begin{multicols}{2}
    \begin{itemize}
        \item Patient data
        \item Electrocardiogram
        \item Cardiac fluoroscopy
        \item Coronary angiograms
    \end{itemize}
\end{multicols}
The category of patient data includes characteristics such as \textit{age}, \textit{sex} or type of chest pain (\textit{cp}), whereas the category electrocardiogram includes various electrocardiographic information obtained during an exercise electrocardiogram like the peak blood pressure (\textit{tpeakbps}). Cardiac fluoroscopy contains all measurements obtained from a cardiac fluoroscopy, a medical measure that allows to see the flow of blood through vessels to evaluate the presence of blockages. An example for an attribute from this category is \textit{ca} which denotes the number of found major vessels. The last category is coronary angiograms, which contains the results of the examination of the same name. The main attribute of interest of this category is \textit{num} because it is the target variable that denotes whether a patient has a heart disease or not. In the raw data, patients without a heart disease are shown with a value of 0 for \textit{num}, while patients with heart disease are shown with a value  $\geq 1$.
An overview of all remaining attributes which were not named here is provided in the code documentation as well as the UCI Machine Learning Repository \citep{janosi1988}. In general, it is assumed that the collected values are of high quality, as they are the result of standardised medical measurement procedures, apart from the individual characteristics described by the patient, such as the location or type of pain. For this reason, it is assumed that a combination of the individual data sets is possible. This is also true because the measurements are not sorted in a certain way, neither in the individual data sets nor in the combined dataset.

The dataset contains the dates of the electrocardiogram and coronary angiograms. These dates are unevenly distributed over a period of seven years. These dates are considered irrelevant because the date of an examination does not affect its outcome as the period is to short to show evolutionary or general health changes in the society. Therefore a time series analysis is not possible with this dataset.

Finally, the distribution of the different variables in the combined dataset was analysed. The target variable was found to be reasonably symmetrically distributed with 495 positive and 404 negative measurements, implying a disease prevalence of about 55.1\% of the examined patients. Non-representative distributions were also found, e.g. \textit{sex} is non-representatively distributed with 78\% males and 22\% females. The same applies to \textit{age}, which is distributed similarly to a normal distribution in the range 28 to 77.

While assuming a high quality of the existing values, it was found that $30,8\%$ of all fields are lacking measurements. This is attributed to the fact that not all universities have carried out all measurements. For example the Cleveland subset lacks the location of chest pain. Some individual attributes in particular have many missing values like history of diabetes (\textit{dm}) with roughly 90\% missing values. Whereby only positive cases may have been entered here, since 95\% of all filled cells show a diabetes disease which again is not representative. Furthermore, when checking whether meanings of attributes and their contained values fit together, some irregularities were observed. For example, the values 0 for cholesterol (\textit{chol}) and blood pressure \textit{testbps} are not compatible with life. Therefore, it is assumed that these values are misreported non-surveys, which are dealt with in the following chapter Preprocessing.
